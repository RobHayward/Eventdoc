%  LaTeX support: latex@mdpi.com 
%  In case you need support, please attach all files that are necessary for compiling as well as the log file, and specify the details of your LaTeX setup (which operating system and LaTeX version / tools you are using).

% You need to save the "mdpi.cls" and "mdpi.bst" files into the same folder as this template file.

%=================================================================
\documentclass[ijfs,article,submit,oneauthor,pdftex,10pt,a4paper]{mdpi} 
%
%--------------------
% Class Options:
%--------------------
% journal
%----------
% Choose between the following MDPI journals:
%ijfs, 
%------
% article
%---------
% The default type of manuscript is article, but can be replaced by: 
% abstract, addendum, article, benchmark, book, bookreview, briefreport, casereport, changes, comment, commentary, communication, conceptpaper, correction, conferenceproceedings, conferencereport, expressionofconcern, meetingreport, creative, datadescriptor, discussion, editorial, essay, erratum, hypothesis, interestingimages, letter, meetingreport, newbookreceived, opinion, obituary, projectreport, reply, reprint, retraction, review, perspective, protocol, shortnote, supfile, technicalnote, viewpoint
% supfile = supplementary materials
% protocol: If you are preparing a "Protocol" paper, please refer to http://www.mdpi.com/journal/mps/instructions for details on its expected structure and content.
%----------
%submit
%----------
% The class option "submit" will be changed to "accept" by the Editorial Office when the paper is accepted. This will only make changes to the frontpage (e.g. the logo of the journal will get visible), the headings, and the copyright information. Also, line numbering will be removed. Journal info and pagination for accepted papers will also be assigned by the Editorial Office.
%------------------
% moreauthors
%------------------
% If there is only one author the class option oneauthor should be used. Otherwise use the class option moreauthors.
%---------
% pdftex
%---------
% The option pdftex is for use with pdfLaTeX. If eps figures are used, remove the option pdftex and use LaTeX and dvi2pdf.

%=================================================================
\firstpage{1} 
\makeatletter 
\setcounter{page}{\@firstpage} 
\makeatother 
\articlenumber{x}
\doinum{10.3390/------}
\pubvolume{xx}
\pubyear{2017}
\copyrightyear{2017}
\externaleditor{Academic Editor: name}
\history{Received: date; Accepted: date; Published: date}

%------------------------------------------------------------------
% The following line should be uncommented if the LaTeX file is uploaded to arXiv.org
%\pdfoutput=1

%=================================================================
% Add packages and commands here. The following packages are loaded in our class file: fontenc, calc, indentfirst, fancyhdr, graphicx, lastpage, ifthen, lineno, float, amsmath, setspace, enumitem, mathpazo, booktabs, titlesec, etoolbox, amsthm, hyphenat, natbib, hyperref, footmisc, geometry, caption, url, mdframed, tabto, soul, multirow, microtype, tikz
\usepackage{rotating}
\usepackage{pdflscape}
\usepackage[flushleft]{threeparttable}
\usepackage{changes}
\setremarkmarkup{(#2)}
%\usepackage[comma, sort&compress]{natbib}% Use the natbib reference package - read up on this to edit the reference style; if you want text (e.g. Smith et al., 2012) for the in-text references (instead of numbers), remove 'numbers' 
%=================================================================
%% Please use the following mathematics environments: Theorem, Lemma, Corollary, Proposition, Characterization, Property, Problem, Example, ExamplesandDefinitions, Hypothesis, Remark, Definition
%% For proofs, please use the proof environment (the amsthm package is loaded by the MDPI class).

%=================================================================
% Full title of the paper (Capitalized)
\Title{Foreign Exchange Speculation: An Event Study}

% Author Orchid ID: enter ID or remove command
%\newcommand{\orcidauthorA}{0000-0000-000-000X} % Add \orcidA{} behind the author's name
%\newcommand{\orcidauthorB}{0000-0000-000-000X} % Add \orcidB{} behind the author's name

% Authors, for the paper (add full first names)
\Author{Rob Hayward $^{1,\dagger,\ddagger}$}

% Authors, for metadata in PDF
\AuthorNames{Rob Hayward}

% Affiliations / Addresses (Add [1] after \address if there is only one affiliation.)
\address{$^{1}$ \quad University of Brighton; rh49@brighton.ac.uk}

% Contact information of the corresponding author
\corres{Correspondence: rh49@brighton.ac.uk; Tel.: +44-1273-642-586}

% Current address and/or shared authorship
\firstnote{Current address: Lewes Road, Brighton BN2 4AT} 
%\secondnote{These authors contributed equally to this work.}
% The commands \thirdnote{} till \eighthnote{} are available for further notes

% Simple summary
%\simplesumm{}

% Abstract (Do not insert blank lines, i.e. \\) 
\abstract{Does speculation facilitate price discovery or instability?  If it is price discovery, it is beneficial and should be encouraged; if it is instability, welfare is enhanced by its reduction. This paper seeks to distinguish between these two characteristics by analysing those times when speculation in the foreign exchange market is most extreme.  A series of event studies are conducted on the extremes of speculative activity and speculative sentiment. If speculation is noise,  extreme sentiment and extreme positions should lead to overshooting and increase risk of subsequent reversals. The finding that speculative extremes do not provide information about subsequent returns implies that speculation is part of the process of price discovery and that efforts to reduce it would reduce the informational efficiency of financial markets.}

% Keywords
\keyword{Foreign Exchange ; Speculation; Event-study.}

% The fields PACS, MSC, and JEL may be left empty or commented out if not applicable
%\PACS{J0101}
%\MSC{}
%\JEL{}

%%%%%%%%%%%%%%%%%%%%%%%%%%%%%%%%%%%%%%%%%%
% Only for the journal Applied Sciences:
%\featuredapplication{Authors are encouraged to provide a concise description of the specific application or a potential application of the work. This section is not mandatory.}
%%%%%%%%%%%%%%%%%%%%%%%%%%%%%%%%%%%%%%%%%%

%%%%%%%%%%%%%%%%%%%%%%%%%%%%%%%%%%%%%%%%%%
% Only for the journal Data:
%\dataset{DOI number or link to the deposited data set in cases where the data set is published or set to be published separately. If the data set is submitted and will be published as a supplement to this paper in the journal Data, this field will be filled by the editors of the journal. In this case, please make sure to submit the data set as a supplement when entering your manuscript into our manuscript editorial system.}

%\datasetlicense{license under which the data set is made available (CC0, CC-BY, CC-BY-SA, CC-BY-NC, etc.)}

%\setcounter{secnumdepth}{4}
%%%%%%%%%%%%%%%%%%%%%%%%%%%%%%%%%%%%%%%%%%

\begin{document}
\section{Introduction}
\added{Speculation can be viewed as stabilising or destabilising.  If speculation is informed, it is a stabilising force, providing liquidity and helping to ensure that markets swiftly find equilibrium.  \citet{KeynesHedge} and \citet{HicksHedge} emphasise the role that speculation plays in facilitating financial market transactions by providing liquidity that allows hedgers and other market participants to swiftly find counterparties without the need for large price concessions.  In this version, speculators offer a service for a fee that is determined by the return on their activity.  \citet{FriedmanPositive} argued that even whew there is a mixture of informed and uninformed speculators, the informed would tend to buy when prices were below intrinsic value and sell when above, making profits, limiting volatility and driving the loss-making uninformed speculators out of business.  However, \citet[p. 101]{Keynes1936} also had a later, more negative view of speculations.  Here he regarded it as a myopic, sentient-driven activity, dominated by the desire to ``beat the gun`` or ``outwit the crowd.  Uninformed speculators may now dominate and \citet{Delong1990noise} provide a model where speculative noise constrains informed traded from benefiting from their knowledge.  Now speculation is an uninformed and destabilising force.  We have therefore two concepts of speculation:  one that is random and disruptive and one that is informed and beneficial. 

%\added{Speculation can be viewed as stabilising or destabilising.} If speculation is based on information it is a stabilising force, providing liquidity and helping to ensure that markets swiftly find equilibrium. \citet{KeynesHedge} and \citet{HicksHedge} \deleted{for example} emphasise the role that speculation plays in facilitating financial market transactions \added{by providing liquidity that allows hedgers and other market participants to swiftly find counterparties without the need for large prices concessions. In this version, speculators} \deleted{Speculators} offer a service for a fee that is determined by the return on their activity.  \added{\citet{FriedmanPositive} argued that even where there is a mixture of informed and uninformed speculators, the informed would tend to buy when prices were below intrinsic value and sell when above, making profits, limiting volatility and driving the loss-making uninformed speculators out of business.}\deleted{While speculators may be trading on information or noise, informed speculators will tend to drive out the uninformed noise-traders according to \citet{FriedmanPositive}.} However, \deleted{the later} \citet[p. 101]{Keynes1936} \added{also had a later, more negative view of speculation.  Here he} regarded it as a myopic, sentiment-driven activity, dominated by the desire to ``beat the gun'' or  ``outwit the crowd''\added{. Uninformed speculators may now dominate} and \citet{Delong1990noise} provide a model \deleted{of cases} where speculative noise \replaced{constrains}{provides a constraint on} informed traders \added{from} benefiting from their knowledge. Now speculation is an uninformed and destabilising force.  \added{We have therefore two concepts of speculation: one that is random and disruptive and one that is informed and beneficial in its proviso of liquidity.

The debate about the relative merit of these two concepts of speculation continues. There has been particular focus on commodities markets and foreign exchange markets.  \citet{Jacksspec} documents a long history of attacks on speculation following price volatility. This has been particularly evident in commodities and foreign exchange markets.  There has been much discussion over the role that speculation played in the boom and bust of commodities prices in the period between 2006 and 2007.  \citet{TangSpec} suggest that this disruption was exacerbated by newly established commodity indices that encouraged speculation.

There have also been investigations of the link between speculation or noise trading and price misalignments or volatility across a range of other asset markets.  \citet{Verma} examine the link between noise trading and equity markets; \citet{Hongjun} argues that speculation is not random and influential in aggregation but that behavioural biases will cause systematic deviations from intrinsic value. Foreign exchange has been a particular focus as a consequence of its liquidity, volatility and the way that speculative attacks appear to have infringed on government sovereignty.  See \citet{Dornbusch1988} in general and \citet{Eichengreen1988} for discussion of the speculative attacks on the European exchange rate system.  \citet{Eichengreen1994} use a set of fundamentals to show that speculative attacks drove exchange rates away from their fundamental value. However, others have made the case that speculative is a key part of the foreign exchange market, providing liquidity and contributing to stability.  See \citet{Mayer1985} and \citet{McKinnon1986}.   

There is no consensus and the the debate enters policy circles with the suggestion that a tax on financial transactions could reduce speculation and thereby limit misalignment and volatility. \citet[pp.158]{TobinTax} and \citet{Tobin1988a} say that such a tax would "throw some sand in the well-greased wheels." of the foreign exchange market. At face value the tax has the advantage of improving market efficiency and raising funds for other welfare-enhancing activity.  However, this is only the case if speculation is uninformed and destabilising.  If it is informed and contributes to stability the tax will come at a cost.} 

This paper evaluates the relative importance of these two concepts of speculation by investigating the relationship between the intensity of speculation and the weight of speculative positions relative to subsequent movement of foreign exchange prices. \added{The foreign exchange market is appropriate this study because it is central to the call for the introduction of a transaction tax as a consequence of it being the largest and most liquid financial market in the world.  The latest survey taken by the Bank for International Settlements (BIS) in April 2016 estimated average daily turnover at \$5.1trn with average daily spot transactions of \$1.7trn See \citet{BISFX2016}.  The cost of trading in this market, when measured in terms of the size of the bid-ask spread, is extremely low. An analysis by \citet{Steely2013} of the bid-ask spreads for USD-JPY, GBP-USD and EUR-USD for the period January 2001 to 2005 finds that the average spread for each is 0.342\%, 0.305\% and 0.326\% respectively.  This would be the cost for a round-trip (buying and selling immediately).This makes it relatively attractive for speculators. 

The intensity of speculation is measured in two ways: speculative sentiment and speculative positions.  Speculative sentiment is identified with the use of option risk-reversal skew.  As these become more extreme it signals that expectations of future value in the foreign exchange options market have diverged significantly from the appropriate forward rate.  The Weight of speculative positions is identified with US regulatory data of the positions held by type of participant in the futures market.  Net long and net short are open positions held by speculators relative to overall activity in the market. See Section \ref{secref:measure} for full data details.}  

The rest of the paper proceeds as follows:  Section Two presents the framework of informed and uninformed speculation; Section Three talks about the measurement of speculation and the event study method; Section Four reviews the evidence; Section Five concludes. 

\section{Informed and uninformed speculation}
Speculation plays an important part in several areas of financial economics.  However, there is no clear, unambiguous theory of speculation.  Speculation is frequently associated with the term trading and the provision of liquidity, but trading is broader as it encompasses financial investment while liquidity provision is really market-making. Speculation and trading can be categorised as being informed or uninformed. Those trading without information are frequently called \emph{noise-traders} or \emph{liquidity traders}.  \added{The first implies some random quality that flow from the lack of information.  However, the psychologists find that the cognitive short-cuts or \emph{heuristics} that are used to make decisions under uncertainty tend to have biases \citet{Kahneman1979Prospect} and numerous studies purport to find these biases in financial asset prices.  For example, \citet{DeBondtOver} find evidence that \emph{representativeness} causes stock prices to overshoot and \citet{Coval2005} identifies evidence of behavioural biases among Chicago Board of Trade proprietary traders.  Liquidity trades are generally those carrying out transactions for reasons other than value.  This can be passive investment funds buying stocks as money is drawn from investors or hedging activity that does not depend on price.    The terms are not fixed and  \citet[p. 1]{GortonNoise} categorise noise-traders as all those trading in markets for ``non-information-based reasons''.  See \citet{Ramiah201589} for a recent review of the noise-trader literature. 

\citet[p.529]{BlackNoise} asserted that the interaction of informed and uninformed activity was essential to the smooth functioning of financial markets. Noise, he argued, was the  ``the arbitrary element in expectations''. Therefore, noise makes financial markets possible but imperfect: the more noise-traders that there are, the more liquidity and the easier it is to trade; the more noise-traders, the higher the level of inefficiency in the market and the more likely that price will be driven from fundamental value. 

Uninformed activity provides a rationale and the opportunity for trading. Unless there is a difference of opinion about value, there is no reason to exchange. Uninformed speculation can create space between price and fundamental value that can be exploited by informed speculators.  This space can encourage informed traders to pay the cost of acquiring information in the \citet{Grossman1980Impossibility} model.  There is equilibrium between the cost of obtaining information and the benefits that are expected to come from the use of the information.  

\added{The foreign exchange market is primarily a dealer-orientated market.  It is largely consistent with the theoretical microstructure models that have been presented by the likes of \citet{Kyle1985Continuous} and \citet{glosten1985bid}. The interaction of informed and uninformed traders drives the price-discovery process: if the uninformed traders are random, the market-maker earns the bid-ask spread; if informed traders arrive they drive the price towards equilibrium.  This study is about short-term movements in foreign exchange prices.  The event windows are between 1 week and 4 weeks in length. While the interpretation of economic fundamentals can drive prices, market positions and liquidity trades will also be important.  Even at longer horizons it is often very difficult to connect exchange rate movements to specific events. \citet{Meese1983Empirical} find that forecasts using exchange rate models cannot beat a random walk while \citet{ChinnModel} updated the study to find very similar results. \citet{Lyons1995Microstructure}, \citet{Evans2002Order} and \citet{rime2010exchange} show that order flow can explain exchange rate movements much more effectively than fundamentals. The initiating or forcing orders drive the price.  Therefore, deliberate speculative should drive the price.  Speculative sentiment changes and speculative positions changes would be part of the price discovery process if informed and would disrupt this process if not. Therefore,if speculators are informed their activity drives exchange rates towards equilibrium and prices should follow a random walk after these extremes of speculative sentiment or position; if speculators are uninformed and their activity can be regarded as noise, the most extreme speculative sentiment and the time when speculators have the greatest weight in the market should correspond with the greatest deviation of prices from fundamental value, when subsequent price reversals are most likely to take place.  }

\section{Measuring speculation}\label{secref:measure}
\added{There are two steps: identify speculation and assess what happens when it becomes extreme.  If speculation is uninformed, extremes should cause a divergence from fundamental value and a price reversal should be anticipated; if informed and part of the price-discovery process, extreme speculation achieves a new equilibrium and the following price performance should be a random walk.  

The measurement of speculative intensity must identify how far speculators believe that the market will move in a particular direction. In the absence of a daily survey of speculators, this study uses a unique series of option\emph{risk-reversal skew} as an measure of speculative sentiment.  The risk-reversal data is very difficult to obtain in an over-the-counter market like foreign exchange where there is no centralised exchange to record and collect information, generally being the property of quoting institutions.  The risk-reversals are the average prices quoted at 10:00 GMT by major banks on the Thomson-Reuters composite option pages (OPT1)  for the period between January 2 1996 and September 6 2002.  The risk-reversals are the spread of implied volatility for a 1 month 25 delta at-the-money call option relative to the equivalent put.  Risk-reversal skew is the relative price of currency put and call options for the same strike price.} The risk-reversal skew is the relationship between a call option and the equivalent put. Equivalence here being measured by the delta or relationship between movement of the underlying price and the movement of the option price.  While the \citet{Black1973Options} model is the basis for valuing options where the price of a non-dividend paying stock trades at a discount to the forward rate equal to the risk free rate, the forward rate in a currency option is determined by the relative interest rate differential between the two currencies.   The \citet{GarmanKohlhagan} adjustment that accounts for the difference in interest payments on the two currencies is used.   The spot discount is determined therefore by the relative interest rate differential for the appropriate maturity.  1 month in this case.  This model allows the market practice of pricing options from the volatility implied by the model.  In other words, the Garman-Kohlhagan is used to extract the volatility that is implied by a specific foreign exchange option price given the strike and other parameters.    

It is common practice to back market expectations out of information drawn from option prices.  See \citet{Malickoption} and \citet{Pochin} for the general approach that combines the implied volatility, risk-reversal and strangles to infer a mixed distribution that can be inferred as the risk-neutral probability density function for a particular asset. See \citet{Gaati} an application on the Yen-dollar exchange rate where expectations are affected by central bank intervention. Using the risk-reversal is a simplified version of that where the skew is used to infer positive and expectations relative to the appropriate forward rate.  See \citet{Eitrheim} and \citet{ECBRR} for previous uses of this data.} 

If the risk-reversal for a 1-month USD/CAD is quoted at 0.15-0.28\% and the implied volatility is 8.50\%, the market-maker would be willing to buy the 1-month 25 delta USD-put-CAD-call at 8.65\% and sell the USD-call-CAD-put at 8.50\%.  The dealer pays 0.15\%.  Alternatively, the dealer would sell a 25 delta call at 8.78\% and buy the USD-call-CAD-put at 8.50\%, earning 0.28\%. See \citet{Global} for a technical review.  Therefore, the relationship between implied volatility for calls and implied volatilities for puts for the same delta and maturity strike is an indication of how much more expensive calls are relative to the equivalent put or as the market bias towards puts over calls and as an indication of market sentiment or related market positioning.  The larger the absolute level of the risk-reversal skew, the greater the relative price and, if the deviation is based on information, subsequent price action should be zero on average; if the deviation is based on noise, this movement should be reversed.  See \citet{Yan2011} and \citet{FengZhangFriesen} for some analysis of implied volatility in the equity market. 

Recording speculative activity in an over-the-counter market like foreign exchange is extremely difficult.  However, there are some exchange-traded currency derivative markets in the US, and participants in these markets are required to report their positions to the US regulators.  The Commodity Futures Trading Commission (CFTC) collect weekly information about the open positions held by private entities in the US derivative markets.   A sub-set of this information is released to the public each week as the \emph{Commitment of Traders Report} (CoT).The \citet{cot} require traders to categorise themselves as being either commercial (C) or non-commercial (NC).  Commercial traders must have some underlying business interest in the security, commodity or instrument that they are trading.  These could be seen as natural hedgers.  Non-commercial accounts are generally considered to be speculators.  See \citet{FuturesSanders} and \citet{FuturesWang} for examples of previous use of this data in this fashion.    The data used run from September 30 1998 when the data started to be released on a weekly basis and continue up to December 31 2008.  The Euro data runs from March 3 1998 to December 31 2008.  There are two gaps in the Swiss Franc data for September 14 2004 and September 21 2004.  The absence of these data seems to be a result of problems with the CFTC database.  The contracts are for EUR 100,000; GBP 62,500; JPY 12,500,000; CHF 100,000 and CAD 100,000.  Two new series are created from the raw data.  These are labelled S1 and S2.  S1 measures net non-commercial positions relative to the total non-commercial positions and S2 measures non-commercial positions relative to open interest. 

\begin{equation}
S1 = \frac{NCL - NCS}{NCL + NCS + NCSP}
\end{equation} 

Where NCL are the non-commercial long positions (holding non-US currency); NCS and non-commercial short positions and NCSP and the non-commercial spreading positions (spread long and short between different contract maturities). This will measure within the speculative group how biased they are towards holding long or short positions. 

\begin{equation}
S2 = \frac{NCL - NCS}{OI}
\end{equation} 

Where NCL are the non-commercial long positions (holding non-US currency); OI is the open interest in the contract. This will measure how biased speculators are to holding long or short positions relative to the whole market.  As such S2 is the preferred measure as it picks up not only direction but also the prominence of speculators relative to other agents. 

S2 is the preferred measure as it 

\added{\subsection{The Event Study}
An event study will be used to identify what happens to foreign exchange prices around the time that speculative sentiment or speculative positions are extreme.  If speculation is uninformed when speculative sentiment or speculative positions are extremely bias to one direction, there should be the greatest divergence from fundamental value and therefore the greatest risk of reversal; if speculation is informed these extremes are part of the adjustment to a new equilibrium.  An event study will analyse the performance of security of asset prices around an event.  In this case the event is the extreme of speculative sentiment or extreme speculative activity. 

\citet{Dolly1933} is credited with the original event study method while \citet{FamaFisherJensenRoll} developed the method to investigate the reaction of equity prices to news of a stock split. An event study will look at the evolution of asset prices in a window around an \emph{event}. One of the major challenges in this process is to disentangle the effect of the event from all the other information that is around at that time.  This problem is usually addressed in two ways: modelling the expected value of the asset and viewing deviations from the expected value as being influenced by the event; using a large sample that, on the condition that there are no confounding factors at that time, would tend to dilute the effect of other information. In this case, the expected value should be the forward rate but given that we are looking at very short time horizons and interest rates for these currencies are relatively low, it is reasonable to take the expected value as zero and to investigate positive or negative exchange rate return.  The samples in this case depend on the how the \emph{extreme} sentiment and positions are defined.  As is seen in Table \ref{tabref:RR1} taking the $5^{th}$ and $95^{th}$ percentile as the extreme for the risk-reversal skew will give around 100 samples for each exchange rate; Table \ref{tabref:SP1} shows that the $95^{th}$ percentile gives just less than 50 samples for the extreme positions. The results do not seem to be sensitive to the definition of extreme. }

\section{Analysis of results}
  Table \ref{tabref:RR1} records the result of the event studies that were carried out on extreme risk reversal.   For each exchange rate the second column distinguishes between extreme high and extreme low risk-reversals, the third  column identifies the number of extreme events that were recorded while the following two columns show the cumulative average returns for the whole event window (CARW) and the cumulative average returns for the event and the period after (CARA).  The windows were chosen to correspond to one day and one week of trading.  
\begin{sidewaystable}
% multirow package is required.   
%pdflandscape is required to facilitate the landscape position. 
\begin{threeparttable}
\caption{Event Study: Cumulative Abnormal Returns and Extreme Risk Reversal}
\begin{tabular}{llccccccccccccc}  
 % \hline
& &\multicolumn{6}{c}{$5^{th}$ and $95^{th}$ percentile} & \multicolumn{1}{c}{} & \multicolumn{6}{c}{$1^{st}$ and $99^{th}$ percentile}\\
 & &\multicolumn{3}{c}{1 day} & \multicolumn{3}{c}{4 day} & \multicolumn{1}{c}{} & \multicolumn{3}{c}{1 day} & \multicolumn{3}{c}{4 day} \\ 
%\cline{4 - 10} \cline{13 - 19}
& &\multicolumn{1}{ c}{N} & \multicolumn{1}{c}{CARW} & \multicolumn{1}{c}{CARA} & \multicolumn{1}{ c}{N} & \multicolumn{1}{c}{CARW} & \multicolumn{1}{c}{CARA} & \multicolumn{1}{c}{} & 
\multicolumn{1}{ c}{N} & \multicolumn{1}{c}{CARW} & \multicolumn{1}{c}{CARA}  
& \multicolumn{1}{ c}{N} & \multicolumn{1}{c}{CARW} & \multicolumn{1}{c}{CARA}  \\   
\hline
\multirow{2}{*}{EUR} 
& Hi &  87 &  0.4103* &0.1586 &87 &0.9442* & 0.1505 &&22 &0.0309 & -0.1748 &22 & 0.5781*& -0.5387*  \\ 
& Lo & 106 &-0.3313* & -0.0933 &106 &-0.0816* &-0.1991 & & 21&0.3040 &0.1820 &21 &0.7935* & 0.4377  \\
\multirow{2}{*}{JPY}
& Hi & 90 &0.4860* &0.1341 &90 &1.5158* &0.2531 & &20 &0.8227* &0.3078* &20 &2.4337* & 0.8284*  \\ 
& Lo & 90 &-0.3168 &0.0945 &90 &-1.3532* &0.1984 & &17 &-1.6990* &-0.9347* &17 &-3.6751* & -1.2556*  \\
\multirow{2}{*}{GBP}
& Hi & 96 &0.3303* &0.1903* &96 &0.8966* &0.2547 & &21 &0.5252* &0.1774 &21 &1.3131* &0.3284  \\ 
& Lo & 97 &-0.1894 &0.0312 &97 &-0.8031* &0.1500 & &24 &-0.2630 &-0.0866 &24 &-0.6660 &0.1888  \\
\multirow{2}{*}{CHF}
& Hi & 80 &0.0065 &-0.0565 &80 &0.5624* &0.1322 & &18 & -0.0022 & -0.1518 &18 &0.4887 &-0.1768  \\ 
& Lo & 75 &-0.3716* & -0.1125& 75 &-1.0071* &-0.0219 & &18 &0.0064 &0.1704 &18 &-0.8661* &0.7631  \\
\multirow{2}{*}{CAD}
& Hi & 113 &-0.0038 &-0.0489 &113 &0.1872 &-0.1951 & & 24 &-0.1081 & -0.0462 &24 &0.3935* & -0.1592  \\ 
& Lo & 97 &-0.1422* & -0.0571 & 97 & -0.2896* & 0.0035& & 26 & -0.2008 & -0.0146 &26 &0.1185 & 0.4560*  \\
\multirow{2}{*}{AUD}
& Hi & 101 &0.2539* &0.0125 &101 &0.8728* & 0.0875 &  &33 &0.5099* &0.0682 &33 & 1.1501*& 0.1046  \\ 
& Lo & 138 &-0.4384* & -0.1711 & 138 & -1.14935* & -0.2714* & &20 &-0.4334 & 0.0563 &20  & -1.7360* & 0.1914  \\
\multirow{2}{*}{EURJPY}
& Hi & 95 &-0.0504 &-0.1103 &95 &-0.0128 &-0.2579 & &23 &-0.2932 &-0.4188* &23 &-0.8203 & -1.0195  \\ 
& Lo & 97 &-0.4746* & -0.1647 &97 &-1.6757* &-0.4595 & &20 &-1.3630* & -0.6976* & 20 & -3.6849* &-1.1999*  \\
\multirow{2}{*}{EURGBP}
& Hi & 91 & 0.1544 &0.0696 &91 &0.5176* &0.0979 & &28 &0.2048 &0.1195 &28 & 0.4146 & -0.0645  \\ 
& Lo & 122 & 0.0451 & 0.0996 &122 &-0.3097* &0.2903* & & 21 &-0.07523 &-0.0780 &21 & -0.4120 &
-0.1290  \\
\multirow{2}{*}{EURCHF}
& Hi & 77 & 0.1183* & 0.0690 &77 & 0.3444* & 0.0749& & 	31 & 0.1482* &0.0739  &31 & 0.4993* 
& 0.1816*  \\ 
& Lo & 94 &-0.0344 & 0.0194 & 94 & -0.1596 &0.1952* & &24 & -0.3626* &-0.1483 & 24 & -0.7573* 
& 0.0875  \\
\hline
\label{tabref:RR1}
\end{tabular}
\begin{tablenotes}
\small 
\item Where Hi means extreme high in the risk-reversal skew and Lo means extreme low in the risk reversal skew; extreme is calculated as being equal or above the $95^{th}$ or $99^{th}$ percentile or equal or below the $5^{th}$ or $1^{st}$ percentile respectively; CARW is the cumulative abnormal return for the whole window, before and after the extreme event, and CARA is the cumulative abnormal return for the period after the event, which is the event day and the window; abnormal return is anything that is different from zero; the asterisk denotes significantly different from zero, where statistical significance is more than 95\% of 1000 means calculated from random bootstrap samples from the events are above or below zero respectively.   
\end{tablenotes}
\end{threeparttable}  
\end{sidewaystable}

As speculators become more extreme in their opinion, exchange rates are driven in the direction of sentiment. Looking down the seventh column (the 4-day CARW) reveals the average cumulative return for the period from four days before the extreme speculative sentiment to the four days after the event.   The extreme highs are those points when the risk-reversal skew is at or above the $95^{th}$ percentile.  For the whole event window (CARW), extreme highs (labelled "Hi") are associated with positive returns and the extreme lows (labelled "Lo") are associated with negative returns. In most instances these are significantly different from zero.  

That sentiment drives price appears evident from the covariance of speculative sentiment with exchange rate prices. This finding is consistent with the evidence of previous research that has used order-flow or sentiment from the likes of \citet{Evans2002Order}, \citet{FuturesSanders} and \citet{FuturesWang}. If speculation is not informed, the extreme is caused by herding or institutional features; if speculation is informed, it is facilitating the absorption of this information into the price. 

Does knowledge about the extreme at event time $t = 0$ provide information about the future returns?  If this speculation is noise, informed traders, such as real money or long-term fundamental accounts, should take advantage of the space that has been created between the price and the fundamental value and a reversal should be seen.  Whether this is the case is recorded in columns 5, 8, 11 and 14 of Table \ref{tabref:RR1}. A reversal is indicated by a significant negative reading for highs or positive reading for lows.   Looking, for example, at the four-day window in column 8, it is clear that the extreme risk reversal skew does not provide any information about future returns.  Of the eighteen cases using the $5^{th}$ and $95^{th}$ percentiles to identify extremely negative and extremely positive sentiment respectively, ten show a continuation (rather than a reversal) that is not significantly different from zero; one shows a significant continuation; five show reversals that are not significant; only two show significant reversal - EURCHF in each case.  For the more extreme event when the $99^{th}$ percentile is used, there is even less clarity.  Just over half the cases with the four-day window show a reversal (8 of 18), but only 2 of these are statistically significant. Exchange rate prices are as likely to go up as down after an extreme and there does not appear to be any information about the future from knowledge of these extremes. Extremes seem to be followed by a random walk as would be the case if the extreme were the result of information being absorbed into the price.  Figures \ref{fig:ES5} and \ref{fig:ES5} show details of a selection of the event studies.  Figure \ref{fig:ES5} is the extreme high ($99^{th}$ percentile) with a 16 day window. There is no evidence of a reversal in any of the currency units.   Figure \ref{fig:ES5} is the extreme low ($10^{th}$ percentile) with a 16 day event window.  Only the CAD shows any evidence of reversal. 

\begin{figure}
\graphicspath{{../Figures/}}
\centering
\includegraphics[scale=0.8]{RRCum16}
\caption{Event Study: Extreme (High $99{th}$ percentile) Risk Reversal Skew and 16 day event window: Risk reversals are the ratio of implied volatility on calls to the equivalent put; extreme high is at or above the $99^{th}$ percentile for the whole range, cumulative returns are the sum of log exchange rate returns. The event day is the day of the extreme risk reversal reading.  If speculation is uninformed, extremes should be followed by reversals; if speculation is informed, extremes are more likely to be followed by a random walk. The solid line is the cumulative mean return from the start of the event window; the dashed lines show the 95\% confidence intervals constructed from 1000 random bootstrap samples of appropriate event window day.}
\label{fig:ES5}
\end{figure}

\begin{figure}
\graphicspath{{../Figures/}}
\centering
\includegraphics[scale=0.8]{RRCum16a}
\caption{Event Study: Extreme (Low $10{th}$ percentile) Risk Reversal Skew and 16 day event window: Risk reversals are the ratio of implied volatility on calls to the equivalent put; extreme low is at or below the $10^{th}$ percentile for the whole range, cumulative returns are the sum of log exchange rate returns. The event day is the day of the extreme risk reversal reading.  If speculation is uninformed, extremes should be followed by reversals; if speculation is informed, extremes are more likely to be followed by a random walk. The solid line is the cumulative mean return from the start of the event window; the dashed lines show the 95\% confidence intervals constructed from 1000 random bootstrap samples of appropriate event window day.}
\label{fig:ES2}
\end{figure}

Table \ref{tabref:SP1} shows the results from the event studies that were carried out using speculative positions as reported to US regulators.  The table works in a similar way to Table \ref{tabref:RR1}.  The first column is the exchange rate, defined as units per US dollar.  The table is then split into two parts.  The first part uses the S1 measure, the net non-commercial (speculative) positions relative to total non-commercial (speculative) positions; the second is S2 as the net non-commercial position relative to open interest or the total outstanding open positions.  Therefore, the first captures the balance of speculative positions while the second captures the balance of speculative positions relative to the whole market.  If speculators become more prevalent or dominant in the market, this will show up in S2 but not S1.  S1 is more sensitive to changes in sentiment and more volatile. Each of these sections is broken into the results for a 2 week window and a 4 week window.  There is a row for extreme high and extreme low which represents extreme long positions or extreme short positions; the first  column identifies the number of cases; CARW is the cumulative abnormal return for the whole event window; CARA is the cumulative abnormal return for the period after the event to the end of the window.     

\begin{sidewaystable}
% multirow package is required.   
%pdflandscape is required to facilitate the landscape position. 
\begin{threeparttable}
\caption{Event Study: Cumulative Abnormal Returns and Extreme Speculative Positions}
\begin{tabular}{llccccccccccccc}	
 % \hline
& &\multicolumn{6}{c}{S1 - Net Long per Speculators} & \multicolumn{1}{c}{} & \multicolumn{6}{c}{S2 -Net Long per Open Positions}\\
 & &\multicolumn{3}{c}{2 week} & \multicolumn{3}{c}{4 week} & \multicolumn{1}{c}{} & \multicolumn{3}{c}{2 week} & \multicolumn{3}{c}{4 week} \\ 
%\cline{4 - 10} \cline{13 - 19}
& &\multicolumn{1}{ c}{N} & \multicolumn{1}{c}{CARW} & \multicolumn{1}{c}{CARA} & \multicolumn{1}{ c}{N} & \multicolumn{1}{c}{CARW} & \multicolumn{1}{c}{CARA} & \multicolumn{1}{c}{} & 
\multicolumn{1}{ c}{N} & \multicolumn{1}{c}{CARW} & \multicolumn{1}{c}{CARA}  
& \multicolumn{1}{ c}{N} & \multicolumn{1}{c}{CARW} & \multicolumn{1}{c}{CARA}  \\   
\hline
\multirow{2}{*}{EUR} 
& Hi &  27 &  2.9415* & 1.3163* &27 & 3.8059* & 1.7341* &&27 & 1.0974* & 0.2896 &27 & 1.7667*& 0.0419  \\ 
& Lo & 27 & -2.4258* & -1.2512* & 27 &-3.9235* &-1.9254* & & 27 & -2.5122* & -1.0607* & 27 & -4.4478* & -1.9285* \\
\multirow{2}{*}{JPY}
& Hi & 43 & 1.9749* & 0.7286* & 43 & 2.8628* & 0.7909 & & 43 & 2.2463* & 1.0554* &43 & 3.7125* & 1.4443*  \\ 
& Lo & 43 & -2.2014* & -1.1312* & 43&-3.2488* & -1.5758* & &43 & -2.0605* & -0.9791* &43  & -2.6865* & -0.8318  \\
\multirow{2}{*}{GBP}
& Hi & 43 & 0.7102 & -0.2116 &43 & 1.0172* & -0.7902* & &43 & 1.2916* & 0.2370 &43 &1.6912* &0.0126  \\ 
& Lo & 43 &-0.6979* &0.2902 &43 &-1.5330* &0.1014 & &43 &-1.8993* &-0.1415 &43 &-2.8830* & -0.0768 \\
\multirow{2}{*}{CHF}
& Hi & 43 & 1.4400* & -0.0988 & 43 & 2.8665* & -0.0993 & & 43 &  2.5260* &  0.5074 & 43 & 3.6060* & 0.2158  \\ 
& Lo & 43 & -2.1816* &  -0.9770* & 43 &-3.3030* & -0.09798* & & 43 & -1.2605* & -0.3070& 43 &-0.1.6021* & 0.0670  \\
\multirow{2}{*}{CAD}
& Hi & 43 & 1.8658* & 1.0195* & 43 & 3.5298* & 1.8319* & & 43 & 2.0160* & 0.8788* &43 & 3.0995* & 0.9921*  \\ 
& Lo & 43 &-1.3238* & -0.5548* & 43 & -1.9151* & -0.7859* & & 43 & -1.2462* & -0.4505*  &43 & -1.9390*  & -0.6195*  \\
\hline
\label{tabref:SP1}
\end{tabular}
\begin{tablenotes}
\small 
\item Where Hi means extreme high, either S1 which is the measure of net long non-commercial (speculative) positions per total of speculators) or S2 which is the measure of net long non-commercial (speculative) positions per open interest (total outstanding open positions); high and low are above the $95^{th}$ percentile or below the $5^{th}$ percentile respectively; CARW is the cumulative abnormal return for the whole window, before and after the extreme event, and CARA is the cumulative abnormal return for the period after the event, which is the event day and the window; abnormal return is anything that is different from zero; the asterisk denotes significantly different from zero, where statistical significance is that more than 95\% of 1000 means calculated from random bootstraps from the extreme readings are above or below zero.   
\end{tablenotes}	
\end{threeparttable}  
\end{sidewaystable}

Once again, there is a very strong association between returns and the activity of speculators.  For the 4-day window and the S1 and S2 measures, every one of the currencies studied, high and low, had a mean cumulative return for the whole of the event window that was in the same direction as sentiment.  When re-sampled a thousand times, more than ninety five percent of the means calculated were greater than zero in all cases but one (GBP).  There is evidence here of momentum and of prices following the movement of speculator positions. 

\begin{figure}
\graphicspath{{../Figures/}}
\centering
\includegraphics[scale=0.8]{FPCum6w}
\caption{Event Study:  Extreme ($90^{th}$ percentile), Non-commercial net long (S1), 6 week window. Non-commercial net per OI (S2) are the net speculative positions (as reported to US regulator the CFTC) relative to all open positions; extreme high is at or above the $90^{th}$ percentile for the whole range, cumulative returns are the sum of log exchange rate returns. The event day is the day of the extreme reading as a measure of extreme speculative activity.  If speculation is uninformed, extremes should be followed by reversals; if speculation is informed, extremes are more likely to be followed by a random walk.}
\label{fig:ES3}
\end{figure}

\begin{figure}
\graphicspath{{../Figures/}}
\centering
\includegraphics[scale=0.8]{FPCum6wa}
\caption{Event Study:  Extreme ($90^{th}$ percentile), Non-commercial net long (S1), 6 week window. Non-commercial net per OI (S2) are the net speculative positions (as reported to US regulator the CFTC) relative to all open positions; extreme high is at or above the $90^{th}$ percentile for the whole range, cumulative returns are the sum of log exchange rate returns. The event day is the day of the extreme reading as a measure of extreme speculative activity.  If speculation is uninformed, extremes should be followed by reversals; if speculation is informed, extremes are more likely to be followed by a random walk.}
\label{fig:ES4}
\end{figure}

There is little evidence that these self-reported speculators are driving prices away from fundamentals with their activity. Sixty percent of the S1 cases still show a significant abnormal return in the direction of the extreme 4 weeks after the extreme has been reached; only thirty percent of the cases show a reversal and in only one of those do ninety five percent of the means calculated after re-sampling remain above zero.  For S2 there is even less clarity, with ninety percent of the cases showing a continuation and forty percent of them showing significant continuation.  The evidence here with speculative positions is the same as that of speculative sentiment.  Prices move in the direction of speculative activity but speculative extremes do not provide information about the future. The price action of the event study is consistent with speculation being informed rather than uninformed.  Figures \ref{fig:ES3} and \ref{fig:ES4} show a selection of event studies that were carried out.  Figure \ref{fig:ES3} is the extreme low ($10^{th}$ percentile) with a 13 week window. There is no evidence of a reversal in any of the currency units.   Figure \ref{fig:ES4} is the extreme high ($90^{th}$ percentile) with a 13 week event window.  Only the GBP shows any evidence of reversal. It is not statistically significant. 


\section{Speculation and foreign exchange returns}
If foreign exchange speculation is uninformed noise, extreme speculation, whether measured by sentiment or weight of activity, will coincide with deviations from fundamental value; if speculation is informed, speculation is part of the process of price discovery and extremes will provide no information about future prices. An event study showed that speculation is associated with price movement: exchange rates move in the direction of speculative sentiment and activity.  However, once extreme \added{speculative sentiment or extreme speculative positions are established}, prices are as likely to continue as to reverse. Speculative sentiment and speculative activity influence foreign exchange prices as indicated by the covariance of speculative measures with returns.  These results show that foreign exchange speculation is an important part of the price discovery process.  The random walk after the extreme is indicative of foreign exchange prices absorbing most of the information that is available.  
  
The results are not a function of the design of the event study.  The event window and the quantiles that are chosen to measure the extreme can be changed.  Alternative measures produced very similar results. Alternative quantiles from 1\% and 99\% to 20\% and 80\% have been assessed; Figures \ref{fig:ES5}, \ref{fig:ES2}, \ref{fig:ES3} and \ref{fig:ES4} have much wider windows and do not find any alternative to the view that extreme speculation is followed by a random walk.  
It is encouraging that two very different measures of speculative activity produce very similar results and that the results hold over different time frames: one week for the option data and one week for the regulatory position data.  It is less clear whether the results would be reproduced across markets for other financial assets.

\added{The foreign exchange market has a particular microstructure that is dominated by over-the-counter transactions in the spot market and this may influence the results.  Further research might look at risk-reversals in financial markets that are traded on electronic exchanges to see if these results can be replicated.  The data on speculative positions are from the exchange traded futures market.} \deleted{The results do not rule out uninformed speculation.  The Friedman effect from the Noise-trader model, whereby the informed speculators take advantage of the space created by price pressure from the uniformed, depends on there being \emph{noise-trader risk} or some limits to arbitrage.  The effect is not evident in this study.}  However, the foreign exchange market is a particularly large and liquid market, other asset classes may have alternative institutional features that mean a different relationship or balance between the informed and uninformed and therefore different results.  

\added{It is possible to dispute the link between risk reversal and market sentiment. It is not only the direct change in expectations about the future that affect  risk-reversals. The weight of positions from hedging activity could be influential. Therefore future research might also investigate the link between these figures and other measures of short-term sentiment if they exist.}  However, a number of studies have found that exchange rates tend to co-move with survey expectations (see for example \citet{FrankelFroot1987} and \citet{Frankel1987Using}. 

\deleted{This is important because there is no unambiguous view of speculative activity and because speculation is an important issues in a number of policy debates.  Most notably, the \emph{Tobin Tax} would put a charge against financial transactions, particularly in foreign exchange.  Part of the argument depends on speculation being uninformed as the reduced activity that is caused by the tax would, it is argued, limit volatility and allow more long-term fundamental evaluations to become prominent.  However, if speculation is informed and part of the price discovery process, this tax may have the opposite effect to what is envisaged.  These results suggest that it could reduce liquidity and increase volatility.} 

 

It is a common refrain that speculation is destabilising and that it will drive prices away from equilibrium.  Testing this assertion is difficult because speculation is difficult to measure and equilibrium cannot be viewed.  However, if speculation is destabilising, when speculators have the most extreme views of the future and when they are most involved in the market should lead to the greatest divergence from equilibrium and the greatest risk that prices will reverse.  Speculation is measured in two ways:  intensity of sentiment and positions.  In each case, foreign exchange prices move with speculative sentiment and speculative positions.  this is consistent with the deliberate orders move prices.  This would be consistent with speculation being part of the process of price discovery.  However, in each case the extreme does not tell us anything about the future direction of the prices. 

Caveats:  futures are small?  risk reversal does not measure expectations, foreign exchange is different,

%=====================================
% References, variant B: external bibliography
%=====================================
\externalbibliography{yes}
\bibliography{myrefs}

%%%%%%%%%%%%%%%%%%%%%%%%%%%%%%%%%%%%%%%%%%
%% optional
%\sampleavailability{Samples of the compounds ...... are available from the authors.}

%%%%%%%%%%%%%%%%%%%%%%%%%%%%%%%%%%%%%%%%%%
\end{document}

