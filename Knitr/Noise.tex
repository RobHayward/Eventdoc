\documentclass[12pt, a4paper, oneside]{article}\usepackage[]{graphicx}\usepackage[]{color}
%% maxwidth is the original width if it is less than linewidth
%% otherwise use linewidth (to make sure the graphics do not exceed the margin)
\makeatletter
\def\maxwidth{ %
  \ifdim\Gin@nat@width>\linewidth
    \linewidth
  \else
    \Gin@nat@width
  \fi
}
\makeatother

\definecolor{fgcolor}{rgb}{0.345, 0.345, 0.345}
\newcommand{\hlnum}[1]{\textcolor[rgb]{0.686,0.059,0.569}{#1}}%
\newcommand{\hlstr}[1]{\textcolor[rgb]{0.192,0.494,0.8}{#1}}%
\newcommand{\hlcom}[1]{\textcolor[rgb]{0.678,0.584,0.686}{\textit{#1}}}%
\newcommand{\hlopt}[1]{\textcolor[rgb]{0,0,0}{#1}}%
\newcommand{\hlstd}[1]{\textcolor[rgb]{0.345,0.345,0.345}{#1}}%
\newcommand{\hlkwa}[1]{\textcolor[rgb]{0.161,0.373,0.58}{\textbf{#1}}}%
\newcommand{\hlkwb}[1]{\textcolor[rgb]{0.69,0.353,0.396}{#1}}%
\newcommand{\hlkwc}[1]{\textcolor[rgb]{0.333,0.667,0.333}{#1}}%
\newcommand{\hlkwd}[1]{\textcolor[rgb]{0.737,0.353,0.396}{\textbf{#1}}}%

\usepackage{framed}
\makeatletter
\newenvironment{kframe}{%
 \def\at@end@of@kframe{}%
 \ifinner\ifhmode%
  \def\at@end@of@kframe{\end{minipage}}%
  \begin{minipage}{\columnwidth}%
 \fi\fi%
 \def\FrameCommand##1{\hskip\@totalleftmargin \hskip-\fboxsep
 \colorbox{shadecolor}{##1}\hskip-\fboxsep
     % There is no \\@totalrightmargin, so:
     \hskip-\linewidth \hskip-\@totalleftmargin \hskip\columnwidth}%
 \MakeFramed {\advance\hsize-\width
   \@totalleftmargin\z@ \linewidth\hsize
   \@setminipage}}%
 {\par\unskip\endMakeFramed%
 \at@end@of@kframe}
\makeatother

\definecolor{shadecolor}{rgb}{.97, .97, .97}
\definecolor{messagecolor}{rgb}{0, 0, 0}
\definecolor{warningcolor}{rgb}{1, 0, 1}
\definecolor{errorcolor}{rgb}{1, 0, 0}
\newenvironment{knitrout}{}{} % an empty environment to be redefined in TeX

\usepackage{alltt} % Paper size, default font size and one-sided paper
%\graphicspath{{./Figures/}} % Specifies the directory where pictures are stored
%\usepackage[dcucite]{harvard}
\usepackage{rotating}
\usepackage{amsmath}
\usepackage{setspace}
\usepackage{pdflscape}
\usepackage[flushleft]{threeparttable}
\usepackage{multirow}
\usepackage{listings}
\usepackage[comma, sort&compress]{natbib}% Use the natbib reference package - read up on this to edit the reference style; if you want text (e.g. Smith et al., 2012) for the in-text references (instead of numbers), remove 'numbers' 
\usepackage{graphicx}
%\bibliographystyle{plainnat}
\bibliographystyle{agsm}
\usepackage[colorlinks = true, citecolor = blue, linkcolor = blue]{hyperref}
%\hypersetup{urlcolor=blue, colorlinks=true} % Colors hyperlinks in blue - change to black if annoying
%\renewcommand[\harvardurl]{URL: \url}
\IfFileExists{upquote.sty}{\usepackage{upquote}}{}
\begin{document}
\title{Noise trader: random walks or forced march?}
\author{Rob Hayward\footnote{University of Brighton Business School, Lewes Road, Brighton, BN2 4AT; Telephone 01273 642586.  rh49@brighton.ac.uk}}
\date{\today}
\maketitle
\begin{abstract}
Noise traders have been characterised as random providers of liquidity, as neutral equivalents of computor-generated random numbers.  Though noise traders may create risk that will deter rational traders from taking advantage of their knowledge of the fundamental value of securities (De-long) and thereby preventing them being driven into extinction as Friedman had asserted (Fridman), their presence  is usually seen as beneficial as they provide liquidity in a neutral fashion.  However, an analysis of noise trading that seeks to measure the weight of noise traders in the market and the strength of their opinin shows that herding is a common feature, highlighting the non-neutral influence that this speculative activity can impose on securities prices. 
\end{abstract}

\section{Introduction}
\section{Literature}
\subsection{Notes}
From "How noise trading..." in the other folder. 
O'Hara and Saar suggest that there is a difference between noise traders and liquidity traders.  Noise traders exert positive effects on liquidity but weaken price reversals. A tax on trading will reduce noise traders but will also reduce the influence of fundamental traders. 

Literature.  Black (1986) highlights the positive effect of noise traders. Shliefer, Summers, DeLong etc identify the limits to arbitrage. The other writings are Battalio, Hatch and Jennings (1997), Harris and Schultz (1998) and Scheinkman and Xiong (2003).  Lowenstein and Willard (2006).  There is considerable debate about the precise role of noise trading in financial markets. There are two strands to the literature on noise trading that emerged in the 1980s:  microstricture literature will use the term "noise traders" and "liquidity traders" interchangeably to identify traders without information.  The motivation of these traders is not identified. Examples of this are Glosten and Milgrom (1985) and Kyle (1985). It is usually assumed to be hedging or liquidity. There is also a \emph{limits-to-arbitrage} literature that uses the term to identify traders that are not movivated by standard explanations of fundamental or liquidity provision. This paper seeks to distinguish between those that trade for liquidty or heging and those that trade because of psychological biases. Fisher Black ``People who trade on noise 
are willing to trade even though from an objective point of view they would be better off 
not trading. Perhaps they think the noise they are trading on is information. Or perhaps 
they just like to trade``.  

In theory, traders may trade on mistaken fundamental information.  Shliefer and Summmer (1990) use momentum.  However, Kaniel, Saar and Titman (2006) find evidence of contrarian strategies. This is associated with short-term speculation. Does the activity drive the price away from fundamental value? Stiglitz (1990) is one of the arguments in favour of a Tobin tax. 

There is a quesiton about whether experimental markets (like this paper) are superior to actual markets where the identification of trader types is difficult.  Theory - Delong, summers, shleifer etc,  Scheinkman and Xiong (2003).  Lowenstein and Willard (2006).  Empirical work:  (see, for example, Garvey and Murphy [2001], Linnainmaa [2003], Barber et. al. [2004], and Kaniel, Saar, and Titman [2006].  

For related papers on securities transaction taxes see Stiglitz [1989], Summers and Summers [1989], 
Amihud and Mendelson [1993], Schwert and Seguin [1993], Umlauf [1993], Subrahmanyam [1998], Dow 
and Rahi [2000], Habermeir and Kirilenko [2003], and Hinman [2003]. 

Interesting difference between supplying and demanding liqudiity.  Limit orders supply liquidity, market orders demand liqudity.  The activity of liquidity traders will tend to change through the course of the session, moving from supplying to demanding liquidity. Noise traders tend to act in the opposite manner.  They tend to supply more liquidity as the session progresses. Black (1986) noise traders put noise into prices.  The effect of noise traders on liquidity: price impact of trades is lower when noise traders are present.  However, this is mostly in the temporary impact on prices rather than the permanent impact. Therefore, noise traders tend to reduce the amount of volatility in prices because market orders do not temporarily have an effect and then reverse.  The presence of noise traders will increase pricing errors. However, this depends on fundametal traders have valuable information. 

There are a number of different ways that noise traders could behave:  activing as SOES bandits (using the NASDAQ small order execution system to make profits) or by using information about the order book; by providing liqudity like a market-maker as a service to liqudiity-traders as in Grossman Miller (1988);  as irrational noise traders, using trading rules or popular models (Shiller 1984, 1990) positive feedback or contrarian.  

Footnote -  There is an empirical evidence that individual investors in various countries trade in a contrarian fashion (e.g., Choe, Kho, and Stulz [1999], Grinblatt and Keloharju [2000, 2001], Jackson [2003], Richards [2005], Kaniel, Saar, Titman [2006]). For experimental evidence consistent with contrarian behavior see 
Bloomfield, Tayler, and Zhou (2997)

Can noise traders be viewed as just random noise.  Dow and Gorton (2006) suggest that this is the case and that this is an outcrop of the rational expectations revolution. Informed traders profit by exploiting the diviations from equilibrium. This is the beneifts of Grossman-Stiglitz. What is the source of this noise?  What are the insitiutions? The importance of noise traders here is that they represent a zero-mean, normally distributed random vaiable.  This is also the description that is provided in Kyle (1985).  Liquidity traders trade due to imbalance in the timing of consumption or from portfolio considerations. Shleifer, Summers and DeLong have trading activity that is affected by sentiment, models or "fads".  This is important.  How the average may not be zero. There is a range of behavioural biases that may account for this (see Barberis and Thaler [2003]).The literaure on hetrogenous traders also focuses on noise traders as trend chasers.  These models (see 
for example Brock and Hommes [1998]; Lux [1995]; Lux and Marchesi [2000]) investigate bubbles and 
crashes, and feature fundamental traders versus noise traders. In these models, it is the noise traders who 
induce the aberrant market behavior.  

\href{http://papers.ssrn.com.ezproxy.liv.ac.uk/sol3/papers.cfm?abstract_id=869827}{Do noise traders move markets?}  There is strong evidence of herding by individual investors. THere is also momentum. There is reversal:  stocks that were heavily bought one year underperform; stocks that were heavily sold, outperform. The opposite in the short run:  heavily bought one week generate large returns the next week. 

\href{http://restud.oxfordjournals.org.ezproxy.liv.ac.uk/content/60/1/1.full.pdf+html}{Campbell and Kyle, Smart money, noise trading and stock market behaviour}. In response to the evidence that the volatilty of stock prices is too great to be a function of discounted expected future earnings (Shiller (1981),  LeRoy and Porter (1981), this paper assesses whether stocks are influnced by exogenous, serially correlated noise. In this model, the noise arises from the interaction of noise traders.  





\section{Methods}
\section{Analysis}
\section{Conclusion}
\end{document}
