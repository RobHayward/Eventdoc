\documentclass[preprint,12pt,authoryear]{elsarticle}
%\graphicspath{{./Figures/}} % Specifies the directory where pictures are stored
%\usepackage[dcucite]{harvard}
% \documentclass[final,1p,times,authoryear]{elsarticle}
%\documentclass[final,1p,times,twocolumn,authoryear]{elsarticle}
%\documentclass[final,3p,times,authoryear]{elsarticle}
%\documentclass[final,3p,times,twocolumn,authoryear]{elsarticle}
%\documentclass[final,5p,times,authoryear]{elsarticle}
%\documentclass[final,5p,times,twocolumn,authoryear]{elsarticle}
\usepackage{rotating}
\usepackage{amsmath}
%\usepackage{setspace}
\usepackage{pdflscape}
\usepackage[flushleft]{threeparttable}
\usepackage{multirow}
\bibliographystyle{agsm}
\usepackage[colorlinks = true, citecolor = blue, linkcolor = blue]{hyperref}
\begin{document}
\begin{frontmatter}
\title{Speculation in foreign exchange: noise or information?}
\author{Rob Hayward\footnote{Brighton Business School, Lewes Road, Brighton, BN2 4AT; Telephone 01273 642586.  rh49@brighton.ac.uk.}}

\begin{abstract}
If foreign exchange prices are driven by speculation that is based on noise rather than information, extreme levels of speculative sentiment and speculative activity should lead to price reversals. Informed speculation is sustainable as it is part of the price discovery process. A unique set of option prices measure speculative sentiment while regulatory positions determine the weight of speculative activity.  An event study shows that extremes of speculative sentiment or speculative activity do not increase the probability of price reversal.  Price activity after the extreme is close to a random walk. This supports the view that speculation is informed and part of the process of price-discovery rather than uninformed noise.      
\end{abstract}
\begin{keyword}
JEL classification: D8; F3; G1. Event study; Information; Noise-trading; Speculation.
\end{keyword}

\end{frontmatter}

\end{document}

